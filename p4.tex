\section{Pregunta N$^{\circ}$4\qquad Alejandro Escobar Mejia}

\begin{frame}
	\begin{enumerate}\setcounter{enumi}{3}
		\item

		      Una familia consta de una madre, un padre y una hija.
		      La suma de las edades actuales de los tres es de $80$ años.
		      Dentro de $22$ años, la edad del hijo sería la mitad que la
		      de la madre.
		      Si el padre es un año mayor que la madre.
		      Determinar la edad de la familia según los siguientes
		      requerimientos:

		      \begin{enumerate}[a)]
			      \item

			            Modele el sistema.

			      \item

			            Resuelva el sistema usando el método de relajación
			            y del descenso más rápido.
		      \end{enumerate}
	\end{enumerate}

	\begin{solution}
		\begin{enumerate}[a)]
			\item

			      Sean las edades de la \alert{madre}, del \alert{padre} y
			      del \alert{hijo}: \alert{$x_{1}$}, \alert{$x_{2}$} y
			      \alert{$x_{3}$}, respectivamente.
			      Entonces, el sistema lineal es

			      \begin{equation*}
				      \systeme{
				      x_{1}+x_{2}+x_{3}=80,
				      -x_{1}+x_{2}=1,
				      -x_{1}+2x_{3}=-22
				      }
			      \end{equation*}

			      Así, la ecuación lineal resulta
			      \begin{equation*}
				      \underbrace{
					      \begin{bmatrix}
						      1  & 1 & 1 \\
						      -1 & 1 & 0 \\
						      -1 & 0 & 2
					      \end{bmatrix}
				      }_{\displaystyle A}
				      \underbrace{
					      \begin{bmatrix}
						      x_{1} \\
						      x_{2} \\
						      x_{3}
					      \end{bmatrix}
				      }_{\displaystyle x}
				      =
				      \underbrace{
					      \begin{bmatrix}
						      80 \\
						      1  \\
						      -22
					      \end{bmatrix}
				      }_{\displaystyle b}.
			      \end{equation*}
		\end{enumerate}
	\end{solution}
\end{frame}

\begin{frame}
	\begin{solution}
		\begin{enumerate}[b)]
			\item

			      .
		\end{enumerate}
	\end{solution}
\end{frame}