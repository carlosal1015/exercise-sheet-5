\section{Pregunta N$^{\circ}$4\qquad Alejandro Escobar Mejia}

\begin{frame}
	\begin{enumerate}\setcounter{enumi}{3}
		\item

		      Una familia consta de una madre, un padre y una hija.
		      La suma de las edades actuales de los tres es de $80$ años.
		      Dentro de $22$ años, la edad del hijo sería la mitad que la
		      de la madre.
		      Si el padre es un año mayor que la madre.
		      Determinar la edad de la familia según los siguientes
		      requerimientos:

		      \begin{enumerate}[a)]
			      \item

			            Modele el sistema.

			      \item

			            Resuelva el sistema usando el método de relajación
			            y del descenso más rápido.
		      \end{enumerate}
	\end{enumerate}

	\begin{solution}
		\begin{enumerate}[a)]
			\item

			      Sean las edades de la \alert{madre}, del \alert{padre} y
			      del \alert{hijo}: \alert{$x_{1}$}, \alert{$x_{2}$} y
			      \alert{$x_{3}$}, respectivamente.
			      Entonces, el sistema lineal es

			      \begin{equation*}
				      \systeme{
				      x_{1}+x_{2}+x_{3}=80,
				      -x_{1}+x_{2}=1,
				      -x_{1}+2x_{3}=-22
				      }
			      \end{equation*}

			      Así, la ecuación lineal resulta
			      \begin{equation*}
				      \underbrace{
					      \begin{bmatrix}
						      1  & 1 & 1 \\
						      -1 & 1 & 0 \\
						      -1 & 0 & 2
					      \end{bmatrix}
				      }_{\displaystyle A}
				      \underbrace{
					      \begin{bmatrix}
						      x_{1} \\
						      x_{2} \\
						      x_{3}
					      \end{bmatrix}
				      }_{\displaystyle x}
				      =
				      \underbrace{
					      \begin{bmatrix}
						      80 \\
						      1  \\
						      -22
					      \end{bmatrix}
				      }_{\displaystyle b}.
			      \end{equation*}
		\end{enumerate}
	\end{solution}
\end{frame}

\begin{frame}
	\begin{solution}
		\begin{enumerate}[b)]
			\item

			      El método llamado sobre-relajación sucesiva (SOR) toma
			      la dirección de Gauss-Seidel hacia la solución y
			      ``sobrepasa'' para tratar de acelerar la convergencia.

			      Sea
			      \begin{math}
				      A=L+D+U
			      \end{math}
                  
                  
                  Para este método primero reescribiremos la matriz A y el vector x de la siguiente manera:

			      \begin{equation*}
				      \underbrace{
					      \begin{bmatrix}
						      1  & 1 & 1 \\
						      -1 & 0 & 1 \\
						      -1 & 2 & 0
					      \end{bmatrix}
				      }_{\displaystyle A}
				      \underbrace{
					      \begin{bmatrix}
						      x_{1} \\
						      x_{3} \\
						      x_{2}
					      \end{bmatrix}
				      }_{\displaystyle x}
				      =
				      \underbrace{
					      \begin{bmatrix}
						      80 \\
						      1  \\
						      -22
					      \end{bmatrix}
				      }_{\displaystyle b}.
			      \end{equation*}
         Ahora multiplicamos por la transpuesta de A a todo el sistema obteniendo el nuevo sistema:
          \begin{equation*}
				      \underbrace{
					      \begin{bmatrix}
						      3  & -1 & 0 \\
						      -1 & 5 & 1 \\
						      0 & 1 & 2
					      \end{bmatrix}
				      }_{\displaystyle A'}
				      \underbrace{
					      \begin{bmatrix}
						      x_{1} \\
						      x_{2} \\
						      x_{3}
					      \end{bmatrix}
				      }_{\displaystyle x}
				      =
				      \underbrace{
					      \begin{bmatrix}
						      101 \\
						      36  \\
						      81
					      \end{bmatrix}
				      }_{\displaystyle b'}.
			      \end{equation*}
         como podemos notar la matriz A' obtenida es tridiagonal, además sus autovalores:
             \begin{math}
				λ_{1}=1.623,
                λ_{2}=2.726,
                λ_{3}=5.651
			\end{math}
         
         podemos verficar entonces que tambien es definida positiva
		\end{enumerate}
	\end{solution}
\end{frame}

\begin{frame}
    \begin{solution}
       De estas condiciones podemos usar el teorema de Ostrowski-Reich que nos dice:
        \begin{equation*}
            w=\frac{2}{1+\sqrt{1-[p[T_{j}]]^{2}}}
        \end{equation*}
        donde \begin{math}
            p[T_{j}]
        \end{math}
        es el radio espectral de la matriz Jacobiana de A'
    entonces procedemos hallarlo:
        \begin{math}
          A=  
        \underbrace{ \begin{bmatrix}
        3 & 0 & 0 \\
        0 & 5 & 0 \\
        0 & 0 & 2 \\
        \end{bmatrix}}_{\displaystyle D}
        -
        \underbrace{\begin{bmatrix}
        0 & 0 & 0 \\
        1 & 0 & 0 \\
        0 & -1 & 0 \\
        \end{bmatrix}}_{\displaystyle L}
        -
        \underbrace{\begin{bmatrix}
        0 & 1 & 0 \\
        0 & 0 & -1 \\
        0 & 0 & 0 \\
        \end{bmatrix}}_{\displaystyle U}
        
        \end{math} 
        \begin{math}
            T_{j}=D^{-1}(L+U)
        =\begin{bmatrix}
            0 & 1/3 & 0 \\
            1/5 & 0 & -1/5 \\
            0 & -1/2 & 0 \\
            \end{bmatrix}
        \end{math}
        \\Entonces obtenemos que
        \begin{math}
            p[T_{j}]=0.408
        \end{math} 
        y que 
        \begin{math}
            w=1.045
        \end{math}
        
    \end{solution}
\end{frame}

\begin{frame}

    \begin{solution}
      Ahora con los datos obtenidos pasamos a usar el metodo SOR:
      \begin{equation}
          (x_{i})^(k)=(1-w) x_{i}^(k-1)+\frac{w}{a_{ii}}
      \end{equation}
    \end{solution}
    
\end{frame}