\section{Pregunta N$^{\circ}$32\qquad Carlos Alonso Aznarán Laos}

\begin{frame}

	\begin{theorem}[Factorización $LU$]
		Si $A\in\mathbb{R}^{n\times n}$ y
		\begin{math}
			\forall k\in\left\{1,\dotsc,n-1\right\}:
			\left|A_{k}\right|\neq 0
		\end{math},
		entonces existen $L\in\mathbb{R}^{n\times n}$ y
		$U\in\mathbb{R}^{n\times n}$, matriz triangular inferior de unos
		en su diagonal y una matriz triangular superior, respectivamente,
		tales que $A=LU$.
		Si $A$ es no singular, entonces la factorización es única y el
		\begin{math}
			\det\left(A\right)=u_{11}\cdots u_{nn}
		\end{math}.

		\

		\begin{equation*}
			\drawmatrix[size=1.2]{A}=
			\drawmatrix[size=1.2, lower, bbox style={fill=blue!10}]{L}\;
			\drawmatrix[size=1.2, upper, bbox style={fill=green!10}]{U}
		\end{equation*}
	\end{theorem}

	\

	\begin{theorem}[Factorización $LDL^{T}$]
		Si $A\in\mathbb{R}^{n\times n}$ es simétrica y
		\begin{math}
			\forall k\in\left\{1,\dotsc,n-1\right\}:
			\left|A_{k}\right|\neq 0
		\end{math},
		entonces existe una matriz triangular inferior de unos en su
		diagonal $L$ y una matriz diagonal $D$ tal que $A=LDL^{T}$.
		La factorización es única.

		\

		\begin{equation*}
			\drawmatrix[size=1.2]{A}=
			\drawmatrix[size=1.2, lower, bbox style={fill=blue!10}]{L}\;
			\drawmatrix[size=1.2, diag, bbox style={fill=red!10}]{D}\;
			\drawmatrix[size=1.2, upper, exponent=T, bbox style={fill=green!10}]{L}
		\end{equation*}
	\end{theorem}
\end{frame}

\begin{frame}
	\begin{theorem}[Factorización de Cholesky]
		Si $A\in\mathbb{R}^{n\times n}$ es simétrica y definida
		positiva, entonces existe una única matriz triangular inferior
		$G\in\mathbb{R}^{n\times n}$ con entradas positivas en la
		diagonal tal que $A=GG^{T}$.

		\

		\begin{equation*}
			\drawmatrix[size=1.2]{A}=
			\drawmatrix[size=1.2, lower, bbox style={fill=blue!10}]{G}\;
			\drawmatrix[size=1.2, upper, exponent=T, bbox style={fill=green!10}]{G}
		\end{equation*}
	\end{theorem}

	\

	\begin{theorem}[Factorización $QR$]
		Si $A\in\mathbb{R}^{m\times n}$, entonces existe una matriz
		ortogonal $Q\in\mathbb{R}^{m\times m}$ y una matriz triangular
		superior $R\in\mathbb{R}^{m\times n}$ tal que $A=QR$.

		\drawmatrixset{%
			black!40, fill=B!30,
			bbox style={draw, black!40}
		}
		\[
			\drawmatrix[width=1.2, height=2]{A}
			= \drawmatrix[width=2, height=2]{Q}
			\;\tikzmark{tmp}\;
			\drawmatrix[width=1.2, height=1.2,
				upper, bbox height=2]{R}
			= \drawmatrix[width=1.2, height=2]{\widehat{Q}} \;
			\drawmatrix[width=1.2, height=1.2, upper]{\widehat{R}}
		\]
		\begin{tikzpicture}[overlay, remember picture]
			\path
			($(pic cs:tmp) +(-1pt, 2pt)$)
			node[draw, dashed, left,
				minimum width=.9cm, minimum height=2.2cm] {};
			\path
			($(pic cs:tmp) +(1pt, -.12)$)
			node[draw, dashed, below right,
				minimum width=1.3cm, minimum height=.9cm] {};
		\end{tikzpicture}

		% \

		% \begin{equation*}
		% 	\drawmatrix[width=1.2, height=2]{A}=
		% 	\drawmatrix[width=2, height=2, fill=yellow!10]{Q}\;
		% 	\drawmatrix[width=1.2, height=1.2, upper, bbox height=2,bbox style={fill=green!10}]{R}
		% \end{equation*}
	\end{theorem}
\end{frame}

\begin{frame}

	\begin{definition}[Espacio fila, espacio columna, rango]
		Sea $A\in\mathbb{R}^{m\times n}$.
		El subespacio de $\mathbb{R}^{1\times n}$ generado por los
		vectores fila de $A$ es el \alert{espacio fila de $A$}.
		El subespacio de $\mathbb{R}^{m}$ generado por los vectores
		columna de $A$ es el \alert{espacio columna de $A$}.
		El \alert{rango de la matriz $A$}, denotado por
		$\operatorname{rang}\left(A\right)$, es la dimensión del espacio
		fila de $A$.
	\end{definition}

	\

	\begin{theorem}
		Si $A\in\mathbb{R}^{m\times n}$, entonces la dimensión del
		espacio fila de $A$ es igual a la dimensión del espacio columna
		de $A$.
	\end{theorem}

	\

	\begin{theorem}[Factorización $QR$ delgada]\label{thm:tinyqr}
		Suponga que $A\in\mathbb{R}^{m\times n}$ tiene rango columna
		completa.
		La factorización $QR$ delgada $A=Q_{1}R_{1}$ es única donde
		$Q_{1}\in\mathbb{R}^{m\times n}$ tiene columnas ortogonales y
		$R_{1}$ es triangular superior con entradas positivas en la
		diagonal.
		Más aún, $R_{1}=G^{T}$ donde $G$ es el factor de Cholesky
		triangular inferior de $A^{T}A$.

		\

		\begin{equation*}
			\drawmatrix[size=1.2]{A}=
			\drawmatrix[size=1.2, fill=yellow!10]{Q_{1}}\;
			\drawmatrix[size=1.2, upper, bbox style={fill=green!10}]{R_{1}}
			\qquad
			\qquad
			\qquad
			\drawmatrix[size=1.2, exponent=T]{A}\;
			\drawmatrix[size=1.2]{A}=
			\drawmatrix[size=1.2, lower, bbox style={fill=blue!10}]{G}\;
			\drawmatrix[size=1.2, upper, exponent=T, bbox style={fill=green!10}]{G}
		\end{equation*}
	\end{theorem}

	% \

	% \begin{example}[Factorización $QR$ delgada]
	% 	Sea
	% 	\begin{math}
	% 		A=
	% 		\begin{bNiceMatrix}
	% 			1 & 0 & 0  \\
	% 			2 & 1 & -1
	% 		\end{bNiceMatrix}
	% 	\end{math}.
	% \end{example}
\end{frame}

\begin{frame}
	\begin{theorem}[Pregunta N$^{\circ}$31]
		Si $R$ es triangular superior e invertible, entonces existe
		una matriz diagonal $D$ con entradas $\pm 1$ en la diagonal tal
		que $R_{1}=DR$ es invertible, triangular superior y tiene
		entradas positivas en la diagonal.

		\

		\begin{equation*}
			\drawmatrix[size=1.2, upper, bbox style={fill=green!10}]{R_{1}}=
			\drawmatrix[size=1.2, diag, bbox style={fill=red!10}]{D}\;
			\drawmatrix[size=1.2, upper, bbox style={fill=green!10}]{R}
		\end{equation*}
	\end{theorem}

	\begin{proof}
		Sean
		\begin{math}
			R=\begin{bNiceMatrix}
				r_{11} & \cdots & r_{1n} \\
				       & \ddots & \vdots \\
				       &        & r_{nn}
			\end{bNiceMatrix},
			D=\begin{bNiceMatrix}
				d_{11} &        &        \\
				       & \ddots &        \\
				       &        & d_{nn}
			\end{bNiceMatrix}
			\in\mathbb{R}^{n\times n}
		\end{math}
		una matriz triangular superior y una matriz diagonal tales que
		\begin{math}
			\forall k\in\left\{1,\dotsc,n\right\}:
			r_{kk}\neq0,
			\left|d_{kk}\right|=1
		\end{math}.
		Entonces,
		\begin{math}
			DR
		\end{math}
		es
		\begin{description}
			\item[invertible]

				porque el
				\begin{math}
					\det\left(DR\right)=
					\det\left(D\right)
					\det\left(R\right)=
					\underbrace{\left(\prod\limits_{i=1}^{n}d_{kk}\right)}_{\neq0}
					\underbrace{\left(\prod\limits_{i=1}^{n}r_{kk}\right)}_{\neq0}
					\neq0
				\end{math}.

			\item[triangular superior]

				\begin{math}
					DR=
					\begin{bNiceMatrix}
						\alert{d_{11}} &        &                \\
						               & \ddots &                \\
						               &        & \alert{d_{nn}}
					\end{bNiceMatrix}
					\begin{bNiceMatrix}
						r_{11} & \cdots & r_{1n} \\
						       & \ddots & \vdots \\
						       &        & r_{nn}
					\end{bNiceMatrix}
					=
					\begin{bmatrix}
						\alert{d_{11}}r_{11} & \cdots & \alert{d_{11}}r_{1n} \\
						                     & \ddots & \vdots               \\
						                     &        & \alert{d_{nn}}r_{nn}
					\end{bmatrix}
				\end{math}
				es triangular superior.
		\end{description}
	\end{proof}
\end{frame}

\begin{frame}
	\begin{description}
		\item[entradas positivas en la diagonal]

			Consideremos la matriz
			\begin{equation*}
				R^{T}R=
				\begin{bNiceMatrix}
					r_{11} &        &        \\
					\vdots & \ddots &        \\
					r_{1n} & \cdots & r_{nn}
				\end{bNiceMatrix}
				\begin{bNiceMatrix}
					r_{11} & \cdots & r_{1n} \\
					       & \ddots & \vdots \\
					       &        & r_{nn}
				\end{bNiceMatrix}=
				\begin{bNiceMatrix}
					r^{2}_{11}   & \cdots & r_{11}r_{1n}                    \\
					\vdots       & \ddots & \vdots                          \\
					r_{11}r_{1n} & \cdots & \sum\limits_{i=1}^{n}r^{2}_{ii}
				\end{bNiceMatrix}.
			\end{equation*}
			Es simétrica porque
			\begin{math}
				\left(R^{T}R\right)^{T}=
				R^{T}{\left(R^{T}\right)}^{T}=
				R^{T}R
			\end{math}
			y es definida positiva porque
			\begin{math}
				\forall k\in\left\{1,\dotsc,n\right\}:
				\left|
				\left(R^{T}R\right)_{k}
				\right|>0
			\end{math}.
			Del teorema de la \alert{factorización de Cholesky},
			existe una única matriz triangular inferior
			\begin{math}
				{\left(DR\right)}^{T}\in\mathbb{R}^{n\times n}
			\end{math}
			con entradas positivas en la diagonal
			tal que
			\begin{math}
				R^{T}R=
					{\left(DR\right)}^{T}
					{\left({\left(DR\right)}^{T}\right)}^{T}
			\end{math}.
			Así, la matriz
			\begin{math}
				DR
			\end{math}
			tiene entradas positivas en la diagonal.
	\end{description}

	\

	\begin{example}[Pregunta N$^{\circ}$31]
		Sea
		\begin{math}
			R=
			\begin{bNiceMatrix}
				1 & 1 \\
				0 & 1
			\end{bNiceMatrix}\in\mathbb{R}^{2\times 2}
		\end{math}
		una matriz triangular superior e invertible porque el
		$\det\left(R\right)=1$.
		Entonces, existe
		\begin{math}
			D=
			\begin{bNiceMatrix}
				\alert{1} & 0         \\
				0         & \alert{1}
			\end{bNiceMatrix}
		\end{math}
		con
		\begin{math}
			\left|a_{11}\right|=
			\left|a_{22}\right|=
			\alert{1}
		\end{math}
		tal que
		\begin{math}
			R_{1}=
			DR=
			\begin{bNiceMatrix}
				1 & 0 \\
				0 & 1
			\end{bNiceMatrix}
			\begin{bNiceMatrix}
				1 & 1 \\
				0 & 1
			\end{bNiceMatrix}=
			\begin{bNiceMatrix}
				\alert{1} & 1         \\
				0         & \alert{1}
			\end{bNiceMatrix}
		\end{math}
		es triangular superior, invertible porque el
		$\det\left(R_{1}\right)=1$ y tiene entradas \alert{positivas}
		en la diagonal.
	\end{example}
\end{frame}

\begin{frame}
	\begin{enumerate}\setcounter{enumi}{31}
		\item

		      Si $A$ tiene vectores columna linealmente independientes,
		      sea $A=QR$ donde $Q$ tiene columnas ortogonales y $R$ es
		      invertible y triangular superior.
		      Mostrar que existe una matriz diagonal $D$ con entradas
		      $\pm 1$ en la diagonal tal que
		      \begin{math}
			      A=
			      \left(QD\right)
			      \left(DR\right)
		      \end{math}.

		      \begin{equation*}
			      \drawmatrix[size=1.2]{A}=
			      \drawmatrix[size=1.2, fill=yellow!10]{Q}\;
			      \drawmatrix[size=1.2, upper, bbox style={fill=green!10}]{R}
			      \qquad
			      \qquad
			      \qquad
			      \drawmatrix[size=1.2]{A}=
			      \left(
			      \drawmatrix[size=1.2, fill=yellow!10]{Q}\;
			      \drawmatrix[size=1.2, diag, bbox style={fill=red!10}]{D}\;
			      \right)
			      \left(
			      \drawmatrix[size=1.2, diag, bbox style={fill=red!10}]{D}\;
			      \drawmatrix[size=1.2, upper, bbox style={fill=green!10}]{R}
			      \right)
		      \end{equation*}
	\end{enumerate}

	\begin{solution}
		% En primer lugar, veamos el siguiente
		Sea
		\begin{math}
			A=
			\begin{bNiceMatrix}[vlines,rules/width=0.2pt]
				c_{1} & \cdots & c_{n}
			\end{bNiceMatrix}
			\in\mathbb{R}^{m\times n}
		\end{math}
		y
		\begin{math}
			\left\{
			c_{1},
			\dotsc,
			c_{n}
			\right\}\subset\mathbb{R}^{m}
		\end{math}
		es un conjunto linealmente independiente.
		Por el teorema de la \alert{factorización QR delgada}, % ~\eqref{thm:tinyqr}
		$A$ posee una factorización (única) $QR$, donde
		$Q\in\mathbb{R}^{m\times n}$ tiene columnas ortogonales y $R$ es triangular
		superior con entradas en la diagonal positivas.

		Además, por la pregunta $31$, como $R$ es triangular superior e
		invertible $\det\left(R\right)>0$, entonces
		\begin{align*}
			A & =
			Q\alert{R}=
			Q\left(
			\alert{DR}
			\right),
			\shortintertext{
			con entradas $\left|{\left(D\right)}_{kk}\right|=1$ en la diagonal
			y $DR$ es invertible, triangular superior y las entradas
			$\left|{\left(DR\right)}_{kk}\right|>0$
			}
			A & =
			Q\left(
			\alert{D^{\frac{1}{2}}}
			\alert{D^{\frac{1}{2}}}
			\right)
			R,
			\shortintertext{
			con entradas
			$\left|{\left(D^{\frac{1}{2}}\right)}_{kk}\right|=1$
			en la diagonal.
			}
		\end{align*}
	\end{solution}
\end{frame}