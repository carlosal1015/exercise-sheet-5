\section{Pregunta N$^{\circ}$32\qquad Carlos Alonso Aznarán Laos}

\begin{frame}

	\begin{theorem}[Factorización $LU$]
		Si $A\in\mathbb{R}^{n\times n}$ y
		\begin{math}
			\forall k\in\left\{1,\dotsc,n-1\right\}:
			\left|A_{k}\right|\neq 0
		\end{math},
		entonces existen $L\in\mathbb{R}^{n\times n}$ y
		$U\in\mathbb{R}^{n\times n}$, matriz triangular inferior de unos
		en su diagonal y una matriz triangular superior, respectivamente,
		tales que $A=LU$.
	\end{theorem}

	\begin{theorem}[Factorización $LDL^{T}$]
		Si $A\in\mathbb{R}^{n\times n}$ es simétrica y
		\begin{math}
			\forall k\in\left\{1,\dotsc,n-1\right\}:
			\left|A_{k}\right|\neq 0
		\end{math},
		entonces existe una matriz triangular inferior de unos en su
		diagonal $L$ y una matriz diagonal $D$ tal que $A=LDL^{T}$.
		La factorización es única.
	\end{theorem}

	\begin{theorem}[Factorización de Cholesky]
		Si $A\in\mathbb{R}^{n\times n}$ es simétrica y definida
		positiva, entonces existe una única
		$G\in\mathbb{R}^{n\times n}$ con entradas en la diagonal
		positivas tal que $A=GG^{T}$.
	\end{theorem}

	\begin{theorem}[Factorización $QR$]
		Si $A\in\mathbb{R}^{m\times n}$, entonces existe una matriz
		ortogonal $Q\in\mathbb{R}^{m\times m}$ y una matriz triangular
		superior $R\in\mathbb{R}^{m\times n}$ tal que $A=QR$.
	\end{theorem}

	\begin{theorem}[Factorización $QR$ delgada]
		Suponga que $A\in\mathbb{R}^{m\times n}$ tiene rango columna
		completa.
		La factorización $QR$ delgada $A=Q_{1}R_{1}$ es única donde
		$Q_{1}\in\mathbb{R}^{m\times n}$ tiene columnas ortogonales y
		$R_{1}$ es triangular superior con entradas diagonal positivas.
		Más aún, $R_{1}=G^{T}$ donde $G$ es el factor triangular
		inferior de $A^{T}A$.
	\end{theorem}
\end{frame}

\begin{frame}
	\begin{enumerate}\setcounter{enumi}{31}
		\item

		      Si $A$ tiene columnas independientes, sea $A=QR$ donde $Q$
		      tiene columnas ortogonales y $R$ es invertible y triangular
		      superior.
		      Mostrar que existe una matriz diagonal $D$ con entradas en
		      la diagonal $\pm 1$ tal que
		      \begin{math}
			      A=
			      \left(QD\right)
			      \left(DR\right)
		      \end{math}.
	\end{enumerate}

	\begin{solution}

		\begin{theorem}[Factorización $QR$]
			Sea $A$ una matriz $m\times n$ con columnas independientes.
			La \alert{factorización $QR$} de $A$ expresa
		\end{theorem}

		Primero veamos el siguiente
		\begin{theorem}
			Si $R$ es triangular superior e invertible, entonces existe
			una matriz diagonal $D$ con entradas en la diagonal $\pm 1$ tal
			que $R_{1}=DR$ es invertible, triangular superior y tiene
			entradas en la diagonal positivas.
		\end{theorem}

		\begin{proof}
			.
		\end{proof}
	\end{solution}
\end{frame}

\begin{frame}
	\begin{solution}

		.
	\end{solution}
\end{frame}