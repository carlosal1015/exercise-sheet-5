\section{Pregunta N$^{\circ}$32\qquad Carlos Alonso Aznarán Laos}

\begin{frame}

	\begin{theorem}[Factorización $LU$]
		Si $A\in\mathbb{R}^{n\times n}$ y
		\begin{math}
			\forall k\in\left\{1,\dotsc,n-1\right\}:
			\left|A_{k}\right|\neq 0
		\end{math},
		entonces existen $L\in\mathbb{R}^{n\times n}$ y
		$U\in\mathbb{R}^{n\times n}$, matriz triangular inferior de unos
		en su diagonal y una matriz triangular superior, respectivamente,
		tales que $A=LU$.
	\end{theorem}

	\

	\begin{theorem}[Factorización $LDL^{T}$]
		Si $A\in\mathbb{R}^{n\times n}$ es simétrica y
		\begin{math}
			\forall k\in\left\{1,\dotsc,n-1\right\}:
			\left|A_{k}\right|\neq 0
		\end{math},
		entonces existe una matriz triangular inferior de unos en su
		diagonal $L$ y una matriz diagonal $D$ tal que $A=LDL^{T}$.
		La factorización es única.
	\end{theorem}

	\

	\begin{theorem}[Factorización de Cholesky]
		Si $A\in\mathbb{R}^{n\times n}$ es simétrica y definida
		positiva, entonces existe una única
		$G\in\mathbb{R}^{n\times n}$ con entradas en la diagonal
		positivas tal que $A=GG^{T}$.
	\end{theorem}

	\

	\begin{theorem}[Factorización $QR$]
		Si $A\in\mathbb{R}^{m\times n}$, entonces existe una matriz
		ortogonal $Q\in\mathbb{R}^{m\times m}$ y una matriz triangular
		superior $R\in\mathbb{R}^{m\times n}$ tal que $A=QR$.
	\end{theorem}

	\

	\begin{theorem}[Factorización $QR$ delgada]\label{thm:tinyqr}
		Suponga que $A\in\mathbb{R}^{m\times n}$ tiene rango columna
		completa.
		La factorización $QR$ delgada $A=Q_{1}R_{1}$ es única donde
		$Q_{1}\in\mathbb{R}^{m\times n}$ tiene columnas ortogonales y
		$R_{1}$ es triangular superior con entradas en la diagonal
		positivas.
		Más aún, $R_{1}=G^{T}$ donde $G$ es el factor triangular
		inferior de $A^{T}A$.
	\end{theorem}
\end{frame}

\begin{frame}

	\begin{definition}[Rango]
		.
	\end{definition}

	\begin{definition}[Columnas independientes]
		.
	\end{definition}

	% Primero veamos el siguiente
	\begin{alertblock}{Pregunta $31$}
		Si $R$ es triangular superior e invertible, entonces existe
		una matriz diagonal $D$ con entradas en la diagonal $\pm 1$ tal
		que $R_{1}=DR$ es invertible, triangular superior y tiene
		entradas en la diagonal positivas.
	\end{alertblock}

	\begin{example}[de la pregunta $31$]
		Sea
		\begin{math}
			R=
			\begin{bmatrix}
				1 & 1 \\
				0 & 1
			\end{bmatrix}\in\mathbb{R}^{2\times 2}
		\end{math}
		una matriz triangular superior e invertible porque el
		$\det\left(R\right)=1$.
		Entonces, existe
		\begin{math}
			D=
			\begin{bmatrix}
				\alert{1} & 0         \\
				0         & \alert{1}
			\end{bmatrix}
		\end{math}
		con
		\begin{math}
			\left|a_{11}\right|=
			\left|a_{22}\right|=
			\alert{1}
		\end{math}
		tal que
		\begin{math}
			R_{1}=
			DR=
			\begin{bmatrix}
				1 & 0 \\
				0 & 1
			\end{bmatrix}
			\begin{bmatrix}
				1 & 1 \\
				0 & 1
			\end{bmatrix}=
			\begin{bmatrix}
				\alert{1} & 1         \\
				0         & \alert{1}
			\end{bmatrix}
		\end{math}
		es triangular superior, invertible porque el
		$\det\left(R_{1}\right)=1$ y las entradas
		de la diagonal son \alert{positivas}.
	\end{example}
\end{frame}

\begin{frame}
	\begin{enumerate}\setcounter{enumi}{31}
		\item

		      Si $A$ tiene vectores columna linealmente independientes,
		      sea $A=QR$ donde $Q$ tiene columnas ortogonales y $R$ es
		      invertible y triangular superior.
		      Mostrar que existe una matriz diagonal $D$ con entradas en
		      la diagonal $\pm 1$ tal que
		      \begin{math}
			      A=
			      \left(QD\right)
			      \left(DR\right)
		      \end{math}.
	\end{enumerate}

	\begin{solution}
		Sea
		\begin{math}
			A=
			\begin{bmatrix}
				c_{1} & \dotsc & c_{n}
			\end{bmatrix}
			\in\mathbb{R}^{m\times n}
		\end{math}
		y
		\begin{math}
			\left\{
			c_{1},
			\dotsc,
			c_{n}
			\right\}\subset\mathbb{R}^{m}
		\end{math}
		es un conjunto linealmente independiente.
		Por el teorema~\eqref{thm:tinyqr}, $A$ posee una factorización
		(única) $QR$, donde $Q\in\mathbb{R}^{m\times n}$ tiene columnas
		ortogonales y $R$ es triangular superior con entradas en la
		diagonal positivas.

		Además, por la pregunta $31$,

		% \begin{theorem}[Factorización $QR$]
		% 	Sea $A$ una matriz $m\times n$ con columnas independientes.
		% 	La \alert{factorización $QR$} de $A$ expresa
		% \end{theorem}
	\end{solution}
\end{frame}

\begin{frame}
	\begin{solution}
		.
	\end{solution}
\end{frame}