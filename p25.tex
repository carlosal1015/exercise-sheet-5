\section{Pregunta N$^{\circ}$25\qquad Brian Alberto Huamán Garcia}

\begin{frame}
	\begin{enumerate}\setcounter{enumi}{24}
		\item

		      Determine el valor de $\alpha$ de modo que la matriz
		      \begin{math}
			      A=
			      \begin{bmatrix}
				      5  & -5     & -6 \\
				      -5 & 3      & -1 \\
				      0  & \alpha & 7
			      \end{bmatrix}
		      \end{math}
		      tenga rango $2$.
		      Luego, determine la solución de mínimos cuadrados cuando
		      \begin{math}
			      b=
			      \begin{bmatrix}
				      -1 \\
				      1  \\
				      -2
			      \end{bmatrix}
		      \end{math}.
	\end{enumerate}

	\begin{solution}
		El rango de $A$ será $2$ si: \\
        EL det($A_{3x3}$) = 0
		\begin{equation*}
			\det\left(A\right)=
			\begin{vmatrix}
				5  & -5     & -6 \\
				-5 & 3      & -1 \\
				0  & \alpha & 7
			\end{vmatrix}=
			5\begin{vmatrix}
				3      & -1 \\
				\alpha & 7
			\end{vmatrix}+
			5\begin{vmatrix}
				-5     & -6 \\
				\alpha & 7
			\end{vmatrix}=
			5\left(21+\alpha\right)+
			5\left(-35+6\alpha\right) = 35 \alpha - 70 = 0 \\
		\end{equation*}
  
    Si Rang(A) = 2: \\
    Entonces: det(A) = 0  y $det(A_{2x2})\neq 0 $ \\
    
    Det(A) = 0 \\
    $35\alpha - 70 = 0 $ \\
    $35 \alpha = 70$ \\
    $\alpha = 2$ \\

    



 
	\end{solution}
\end{frame}

\begin{frame}
	\begin{solution}

    $Det(A_{2x2}) \neq 0$ 
    
    \begin{vmatrix}
		5     & -5 \\
		-5    &  3
	\end{vmatrix}= 40 \\

    \begin{vmatrix}
		5     & -6 \\
		3     & -1 
	\end{vmatrix}= 23 \\

    \begin{vmatrix}
		-5    & 3 \\
		0     & 2 
	\end{vmatrix}= -10 \\

    \begin{vmatrix}
		3     & -1 \\
		2     & 7 
	\end{vmatrix}= 23 \\

    Entonces la matriz A sería: \\

    \begin{math}
		A=
		\begin{bmatrix}
			5  & -5     & -6 \\
			-5 & 3      & -1 \\
			0  & 2      & 7
		\end{bmatrix}
	\end{math} \\

    Entonces tenemos que: \\
    fila 3 = combinación lineal de f1 y f2 \\
    $f_{3}= \alpha f_{1} + \beta f_{2}$ \\

    Tenemos que: \\
    $Ax=b$ \\
    $A^{T}.A.x=A^{T}.b$\\
    $(A^{T}.A)^{-1}(A^{T}.A).x = (A^{T}.A)^{-1} A^{T}.b $ \\
    $x = (A^{T}.A)^{-1} . A^{T} . b$
    
    \end{solution}
\end{frame}

\begin{frame}
	\begin{solution}

    $A^{T}.A=$
    \begin{vmatrix}
		50  &  -40  & -25 \\
		-40 &   38  &  41 \\
	  -25 &   41  &  86 \\
	\end{vmatrix} \\

    Hacemos matriz escalonada por Gauss-Jordan: \\
    $A^{T}.A = $
    \begin{vmatrix}
		1 &  0  & 23/10 \\
		0 &  1  & 7/2 \\
	  0 &  0  &  0 \\
	\end{vmatrix} \\

    $Det(A^{T}.A) = 0$\\
    Entonces: \\
    $(A^{T}.A)^{-1}$ no existe \\
    Entonces: \\
    $x = (A^{T}.A)^{-1}.A^{T}.b $ no existe \\
    Entonces: \\
    x = Tiene infinitas soluciones \\
    Entonces hacemos la matriz aumentada: \\
    
    \begin{equation*}
        
        \begin{bmatrix}
	       A^{T}.A  &  \mid & A^{T}.b \\
	  \end{bmatrix} = \\
    
        \begin{bmatrix}
			1  &     & 23/10 & \mid & -10 \\
		  0  & 1   & 7/2   & \mid &  4   \\
			0  & 0   & 0     & \mid & 4
		\end{bmatrix} \\
  
    \end{equation*}
    
    \end{solution}
\end{frame}

\begin{frame}
    \begin{solution}
     Tenemos que: \\
    $x_{3}=t$ \\
    $x_{2}+\frac{7}{2}t=4$ \\
    $x_{2} = 4 - \frac{7}{2}t$ \\
    $x_{1}+\frac{23}{10}t = -10$ \\
    $x_{1}= -10 - \frac{23}{10}t$ \\

    \begin{math}
    
    \widehat{x} =
        \begin{bmatrix}
		  -10 - \frac{23}{10}t    \\
		  4 - \frac{7}{2}t   \\
			t  
		\end{bmatrix} \\
    
    \end{math}

    Entonces el mínimo cuadrado es: \\
    \begin{math}
        Min\left \| A\widehat{x}-b \right \| ^ 2 \\
    \end{math}

    Entonces tenemos que: \\
        A$\widehat{x}$ - b= 
        \begin{bmatrix}
		  -70  \\
		  62   \\
			8  
		\end{bmatrix} - 

        \begin{bmatrix}
		  -1  \\
		  1   \\
			2  
		\end{bmatrix} \\
  
    \end{solution}
\end{frame} \\

\begin{frame}
    \begin{solution}
        \begin{math}
            A\widehat{x}-b =
            \begin{Vmatrix}
		  -71  \\
		  61   \\
		  6  
		  \end{Vmatrix} ^2 \\
        \end{math}

    \end{solution}    
\end{frame}