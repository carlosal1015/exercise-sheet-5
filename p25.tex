\section{Pregunta N$^{\circ}$25\qquad Brian Alberto Huamán Garcia}

\begin{frame}
	\begin{enumerate}\setcounter{enumi}{24}
		\item

		      Determine el valor de $\alpha$ de modo que la matriz
		      \begin{math}
			      A=
			      \begin{bmatrix}
				      5  & -5     & -6 \\
				      -5 & 3      & -1 \\
				      0  & \alpha & 7
			      \end{bmatrix}
		      \end{math}
		      tenga rango $2$.
		      Luego, determine la solución de mínimos cuadrados cuando
		      \begin{math}
			      b=
			      \begin{bmatrix}
				      -1 \\
				      1  \\
				      -2
			      \end{bmatrix}
		      \end{math}.
	\end{enumerate}

	\begin{solution}
     Primero hallemos el rango de la Matriz:\\
     Probemos Rango = 3  ( Por Método de Determinantes): \\
        \begin{math}
			det(A)=
			\begin{vmatrix}
			5  & -5     & -6 \\
			-5 & 3      & -1 \\
			0  & \alpha & 7
			\end{vmatrix}
		\end{math} = 
            5 
            \begin{vmatrix}
		  3      & -1 \\
			\alpha & 7
			\end{vmatrix} +5
            \begin{vmatrix}
		  -5     & -6 \\
			\alpha & 7
		.   \end{vmatrix} \\
        \hspace{5cm} = 5 (21 + \alpha) + 5 (-35 + 6\alpha)
	\end{solution}
\end{frame}

\begin{frame}
	\begin{solution}

		.
	\end{solution}
\end{frame}